%%%%%%%%%%%%%%%%%%%%%%%%%%%%%%%%%%%%%%%%%
% LaTeX Template
% Version 1.0 (26 de setembro de 2025)
% Author: Enzo Libório Fraga (enzoliborio@academico.ufs.br)
% License: CC BY 4.0
% Disponível em: https://github.com/enzoliborio/Notas_de_Aulas-template
%%%%%%%%%%%%%%%%%%%%%%%%%%%%%%%%%%%%%%%%%

\documentclass[12pt,openright,twoside,a4paper,english,brazil]{book}

\usepackage[utf8]{inputenc}	% Codificacao do documento (conversão automática dos acentos)
\usepackage[T1]{fontenc}		% Selecao de codigos de fonte.
\usepackage{lmodern}	        % Usa a fonte Latin Modern


%-----------------------------------------------------------
% BIBLIOGRAFIA
%-----------------------------------------------------------
% \usepackage[style=abnt, backref=false, language=brazil]{biblatex} % Carrega o estilo ABNT
\usepackage{biblatex}
\usepackage[brazilian]{babel} % Suporte ao idioma brasileiro
\addbibresource{Referencias.bib}


% ---------------------------------------
% PACOTES ESSENCIAIS
% ---------------------------------------
% CORES
\usepackage[table,xcdraw,dvipsnames]{xcolor}
\definecolor{mintedgreen}{HTML}{155C12}
\definecolor{mintedgreenNovo}{HTML}{0B610B}

\usepackage{graphicx} % imagens
\usepackage{float} % para \begin{figure}[H]
% \usepackage{caption} % legenda de figuras\tabelas
\usepackage{subcaption} % para usar o ambiente subfigure/subcaption

\captionsetup[figure]{font={footnotesize}} % ALTERAÇÃO: fonte menor na legenda de figuras
\captionsetup[table]{font={footnotesize}} % ALTERAÇÃO: fonte menor na legenda de tabelas
% \usepackage{subfigure} %para subfiguras (Figura 1a, 1b, ...)

\usepackage{booktabs} % Para \toprule, \midrule, \bottomrule

\usepackage{pdfpages} % CAPA
\pdfmajorversion=1
\pdfminorversion=7

\usepackage{emptypage} % Garante que páginas em branco sejam realmente em branco
\usepackage{appendix} % Apêndice

\usepackage{indentfirst} % Indenta o primeiro parágrafo
\usepackage{microtype} % para melhorias de justificação
\usepackage{setspace} % para \begin{spacing} [...] \end{spacing}

\usepackage{enumerate} % listas enumeradas
\usepackage{enumitem} % listas pontilhadas

\usepackage{titlesec} % Personalização de títulos
% O tamanho do parágrafo é dado por:
\setlength{\parindent}{1.3cm}
% Controle do espaçamento entre um parágrafo e outro:
\setlength{\parskip}{0.2cm}

% escita matemática:
\usepackage{amsthm}
\usepackage{amsmath}
\usepackage{amssymb,amsfonts,amsthm}
\usepackage{bigints} % integrais maiores

\usepackage{siunitx} % para \celsius

\usepackage{url} % usar \url{}
\urlstyle{same}

% \usepackage{chngcntr} % Pacote necessário para redefinir os contadores
% \counterwithin{figure}{section} % Faz as figuras serem numeradas por subsection

% \usepackage{extsizes} % Suporte para tamanhos de fonte maiores

\usepackage{lipsum} % para texto de exemplo

\usepackage{graphics}


% ---------------------------------------
% AMBIENTE \begin{blocks}
% ---------------------------------------
\usepackage[most,theorems]{tcolorbox}

\newtcolorbox{blocks}[1]{
	colback=black!5!white,     % Cor de fundo do corpo do texto
	colframe=black!70!white,   % Cor da fundo do título
	fonttitle=\bfseries,
	title={#1},
	enhanced,                  % Melhora a aparência
	boxrule=0.4pt,             % Espessura da borda
	arc=4pt                    % Arredondamento dos cantos
}

\newtcbtheorem[auto counter, number within=chapter]{teorema}{Teorema}{%
    colback=mintedgreen!5!white,    % Cor de fundo do corpo do texto
	colframe=mintedgreen!70!white,  % Cor da fundo do título
	fonttitle=\bfseries,
    title={#1},
	enhanced,                      % Melhora a aparência
	boxrule=0.4pt,                 % Espessura da borda
	arc=4pt,                       % Arredondamento dos cantos
}{teorema} % Prefixo para labels (ex: \label{teorema:meu_teorema})

\theoremstyle{definition} % Texto normal, título em negrito
\newtheorem{exemplo}{\( \blacksquare \) Exemplo}[chapter] % para exemplos numerados pelo capítulo





% ---------------------------------------
% HYPER LINKS
% ---------------------------------------
\usepackage{hyperref}
\hypersetup{
	%pagebackref=true,
	colorlinks=true,       		% false: boxed links; true: colored links
	linkcolor=black,          	% color of internal links
	citecolor=OliveGreen,            % color of links to bibliography
	filecolor=magenta,      	% color of file links
	urlcolor=OliveGreen,
	bookmarksdepth=4
}

% ---------------------------------------
% GRÁFICOS
% ---------------------------------------
\usepackage{mathrsfs}
\usetikzlibrary{arrows}
\usepackage{tikz}
\usepackage{pgfplots}
\usetikzlibrary{
	angles, % Para desenhar o ângulo theta
	quotes, % Para colocar o rótulo no ângulo
	decorations.pathreplacing % Para desenhar a chave sob o eixo x
}

%usepackage[mode=buildnew]{standalone}
%\usepackage{import}
\pgfplotsset{compat=1.9}
%\usepgfplotslibrary{patchplots}
%\usepgfplotslibrary{external}
%\usetikzlibrary{external}
%\tikzset{
	%external/system call={xelatex \tikzexternalcheckshellescape -halt-on-error -interaction=batchmode -jobname "\image" "\texsource"}}
%\tikzexternalize[prefix=tikz/]


% ---------------------------------------
% PARA CÓDIGOS
% ---------------------------------------
\usepackage{fvextra} % Necessário para 'linenos' funcionar
\usepackage{csquotes} % adicionar DEPOIS de \usepackage{fvextra}
\MakeOuterQuote{"} % para usar aspas "

\usepackage{minted}
\setminted{
    %style=friendly,     % 4. Muda o estilo para um sem itálico nos comentários
    %fontsize=\small,
    linenos,            % Habilita números de linha
    breaklines,
    frame=single,
    xleftmargin=1em,    % 1. Empurra tudo para dentro da margem
    numbersep=5pt       % 2. Aproxima os números do código
}

% \usepackage{listings}
% \renewcommand{\listingscaption}{Código}

\usepackage{newfloat}
\DeclareFloatingEnvironment[fileext=loc, within=chapter]{codigo}
\floatname{codigo}{Código} % nome na legenda
\captionsetup[codigo]{skip=-3mm} % distância entre código e legenda

% Título
\newcommand{\titulo}{Álgebra Linear}
\newcommand{\subtitulo}{Notas de aulas}

% Autor
\newcommand{\autorA}{Enzo Libório Fraga}
\newcommand{\emailAutorA}{enzoliborio@academico.ufs.br}

% Dados da Instituição ao qual pertence
\newcommand{\instituicao}{Universidade Federal de Sergipe}
\newcommand{\campus}{Cidade Universitária Prof. José Aloísio de Campus}
\newcommand{\centro}{Centro de Ciêncas Exatas e Tecnologia}
\newcommand{\departamento}{Departamento de Computação}
\newcommand{\disciplina}{Álgebra Linear}

% Nome do Professor
\newcommand{\professor}{Dr\textsuperscript{a}. }
\newcommand{\emailProfessor}{}

% Local
\newcommand{\cidade}{Aracaju}
\newcommand{\estado}{SE}

% Datas
\newcommand{\dia}{dia}
\newcommand{\mes}{mês}
\newcommand{\ano}{ano}
% ---------------------------------------------------
% COMANDOS
% ---------------------------------------------------


% ---------------------------------------------------
% QUADRADOS NO FIM DA RESOLUÇÃO OU PROVA MATEMÁTICA
% ---------------------------------------------------
\newcommand{\fimresolucao}{%
	\ifmmode%
	   \tag*{$\square$}% Modo matemático (alinhado à direita como \tag*)
	\else%
	   \hfill $\square$% Modo texto (alinhado à direita com \hfill)
	\fi%
}

\newcommand{\fimprova}{%
	\ifmmode%
	   \tag*{$\blacksquare$}% Modo matemático
	\else%
	   \hfill $\blacksquare$% Modo texto
	\fi%
}



% ---------------------------------------------------
% Numeração das definições, exemplos, observações e teoremas 
% ---------------------------------------------------
% \theoremstyle{definition} % Texto normal, título em negrito

% \newtheorem{definicao}{Definição}[chapter] % para definições numeradas pelo capítulo

% \newtheorem{observacao}{Observação}[chapter] % para observações numeradas pelo capítulo

% \newtheorem{teorema}{Teorema}[chapter] % para teoremas numerados pela pelo capítulo

% % não é usado dentro do ambiente blocks *veja o capítulo 1
% \newtheorem{exemplo}{\( \blacksquare \) Exemplo}[chapter] % para exemplos numerados pelo capítulo


% Configuração das margens
\usepackage[top=3cm, bottom=2cm, left=3cm, right=2cm]{geometry}


% -------------------------------------------------------
% DOCUMENTO
% -------------------------------------------------------
\begin{document}
	
	% -------------------------------------------------------
	% PRÉ-TEXTUAL
    % -------------------------------------------------------
    % Aqui estão os elementos anteriores ao corpo do texto.
    % Estão pré-adicionados somente o Resumo e Prefácio, mas pode-se adicionar mais, como dedicatória, lista de figuras, etc
    % Lembre de criar, para cada um novo, um documento .tex na pasta "Pre_textual"
	% -------------------------------------------------------
	\frontmatter
	
	% ---- CAPA ----
	\thispagestyle{empty}
	% ---------------------------------------------------
% se sua capa for um PDF, descomente a linha 5
% e substituia <local> pelo endereço do PDF
% ---------------------------------------------------
% \includepdf[pages={1}, scale=1, offset=0mm 0mm, fitpaper=true, pagecommand={}]{<local>.pdf}


% ---------------------------------------------------
% se sua capa somente seguir os dados:
% ---------------------------------------------------
\title{Notas de Aulas em \LaTeX}
\author{Enzo Libório Fraga}
\date{Setembro de 2025}

\maketitle
	
	%\include{Pre_textual/Dedicatoria.tex} % a dedicatória, por exemplo, vem anterior ao sumário
	
	% ---- SUMÁRIO ----
	\begin{spacing}{0.8}
		\pdfbookmark[0]{\contentsname}{toc}
		\tableofcontents
	\end{spacing}
	
	% ---- RESUMO e PREFÁCIO ----
	\chapter*{Resumo}
\markboth{RESUMO}{}
\addcontentsline{toc}{chapter}{Resumo} % Adiciona "Resumo" ao sumário

\begin{onehalfspace} %espaçamento de 1.5
	Estas notas de aulas foram elaboradas por <nome>\footnote{Graduando em Engenharia de Computação na Universidade Federal de Sergipe (UFS), campus São Cristóvão. Currículo Lattes: \url{<lattes>}. E-mail: \href{mailto:<email>}{<email>}.} com base no livro \textit{<livro>}, de <autor> \cite{autorGenerico}, e nas anotações das aulas da disciplina de <disciplina>, integrante da grade curricular do curso de Engenharia de Computação em 2025, ministrada pelo Professor Dr. <professor>\footnote{Professor do Departamento de <departamento> da Universidade Federal de Sergipe (UFS), campus São Cristóvão. Currículo Lattes: \url{<lattes>}. E-mail: \href{mailto:<email>}{<email>}.}.
\end{onehalfspace}

% SUBSTITUA PELO SEU NOME E SUAS INFORMAÇÕES PESSOAIS, O(A) PROFESSORA(A) TAMBÉM!
	\chapter*{Prefácio}
\markboth{PREFÁCIO}{}
\addcontentsline{toc}{chapter}{Prefácio} % Adiciona ao sumário

\begin{onehalfspace} % espaçamento de 1.5
	\lipsum[1-6] % apague essa linha e escreva o prefácio
\end{onehalfspace}

\begin{flushright}
	Sergipe, setembro de 2025
	
	Enzo Libório Fraga
\end{flushright}
	
	
	
	% ----------------------------------------
	% Inicia o corpo principal do texto
	% ----------------------------------------
	\mainmatter
	
	% ---- CAPÍTULOS ----
    % ---------------------------------------------------
% CAPÍTULO
% ---------------------------------------------------
\chapter{Introdução ao modelo \textit{Notas de Aulas} em \LaTeX}

Aqui estão alguns exemplos de estruturas pré-definidas do \LaTeX\ ou criadas por mim para melhorar a organização. Para visualizar a escrita do código em latex, acesse o arquivo \verb|00_Introducao_ao_template.tex|.


% ---------------------------------------------------
% SEÇÃO
% ---------------------------------------------------
\section{Texto}

\subsection{Texto colorido}

{\textcolor{OliveGreen}{Lorem ipsum dolor em \texttt{OliveGreen}}}

{\textcolor{Red}{Lorem ipsum dolor em \texttt{Red}}}

{\textcolor{Blue}{Lorem ipsum dolor em \texttt{Blue}}}


% ---------------------------------------------------
\subsection{Listas numeradas}

Lorem ipsum dolor sit amet, consectetur adipiscing elit:
\begin{enumerate}
    \item Lorem ipsum dolor sit amet.
    \item Lorem ipsum dolor sit amet.
\end{enumerate}


% ---------------------------------------------------
\subsection{Listas ecom marcadores}

Lorem ipsum dolor sit amet, consectetur adipiscing elit:
\begin{itemize}
    \item Lorem ipsum dolor sit amet.
    \item Lorem ipsum dolor sit amet.
\end{itemize}


% \newpage


% ---------------------------------------------------
% SEÇÃO
% ---------------------------------------------------
\section{Ambientes matemáticos}

\begin{itemize}
    \item Exemplo de ambiente \texttt{equation} (equações numeradas automaticamente):
\end{itemize}
\begin{equation}
    \lim_{n\to \infty} \sum_{i=1}^{n} f(x_i)\,\Delta x 
    = \int_{a}^{b} f(x)\,dx = F(b) - F(a)
\end{equation}


\begin{itemize}
    \item Equação em exibição sem numeração:
\end{itemize}
\[
\nabla \cdot \vec{E} = \frac{\rho}{\varepsilon_0}
\]

\begin{itemize}
    \item Equações que precisam ficar alinhadas no "=", por exemplo:
\end{itemize}

Equações de Navier-Stokes Forma expandida (3D):
\begin{align*}
    \rho\left(\frac{\partial u}{\partial t} + u\frac{\partial u}{\partial x} + v\frac{\partial u}{\partial y} + w\frac{\partial u}{\partial z}\right) &= -\frac{\partial p}{\partial x} + \mu\left(\frac{\partial^2 u}{\partial x^2} + \frac{\partial^2 u}{\partial y^2} + \frac{\partial^2 u}{\partial z^2}\right) + f_x \\[0.3cm]
    \rho\left(\frac{\partial v}{\partial t} + u\frac{\partial v}{\partial x} + v\frac{\partial v}{\partial y} + w\frac{\partial v}{\partial z}\right) &= -\frac{\partial p}{\partial y} + \mu\left(\frac{\partial^2 v}{\partial x^2} + \frac{\partial^2 v}{\partial y^2} + \frac{\partial^2 v}{\partial z^2}\right) + f_y \\[0.3cm]
    \rho\left(\frac{\partial w}{\partial t} + u\frac{\partial w}{\partial x} + v\frac{\partial w}{\partial y} + w\frac{\partial w}{\partial z}\right) &= -\frac{\partial p}{\partial z} + \mu\left(\frac{\partial^2 w}{\partial x^2} + \frac{\partial^2 w}{\partial y^2} + \frac{\partial^2 w}{\partial z^2}\right) + f_z
\end{align*}
onde $\mathbf{v} = (u,v,w)$ é o campo de velocidade, $p$ é a pressão, $\rho$ é a densidade, $\mu$ é a viscosidade dinâmica e $\mathbf{f}$ representa forças externas.

\begin{itemize}
    \item Texto de exemplo com equação em linha: \( E = mc^2 \) ou também $a^2 + b^2 = c^2$.
\end{itemize}

\begin{itemize}
    \item Equação dentro de uma caixa:
\end{itemize}
\begin{equation}
    \boxed{ \nabla f(x,y,z) = \lambda\ \nabla g(x,y,z) }
\end{equation}



% \newpage


% ---------------------------------------------------
% SEÇÃO
% ---------------------------------------------------
\section{Códigos de programação}

% Exemplos com o pacote minted
% (necessário compilar com -shell-escape)

Para escrever um código de programação usando o ambiente \texttt{codigo} e \texttt{minted}:

\begin{codigo}[H]
    \begin{verbatim}
\begin{codigo}[H]
\begin{minted}{<linguagem>}
<código>
\end{minted}
\caption{<legenda>}
\label{codigo:<label>}
\end{codigo}
    \end{verbatim}
    
    \caption{Código-fonte em LaTeX para gerar um bloco de código Java. Fonte: Autor.}
    \label{cod:latex-source-verbatim}
\end{codigo}

% ---------------------------------------------------
\subsection{Linguagem Java}

\begin{codigo}[H]
\begin{minted}{java}
public class Main {
    public static void main(String[] args) {
        System.out.println("Olá, mundo!");
    }
}
\end{minted}
\caption{Código em Java usando \texttt{minted}. Fonte: Autor.}
\label{cod:java}
\end{codigo}

% ---------------------------------------------------
\subsection{Linguagem C}

\begin{codigo}[H]
\begin{minted}{c}
#include <stdio.h>

int main() {
    printf("Olá, mundo!\n");
    return 0;
}
\end{minted}
\caption{Código em C usando \texttt{minted}. Fonte: Autor.}
\label{cod:c}
\end{codigo}

% ---------------------------------------------------
\subsection{Linguagem Python}

\begin{codigo}[H]
\begin{minted}{python}
def soma(a, b):
    return a + b

print("Resultado:", soma(2, 3))
\end{minted}
\caption{Código em Python usando \texttt{minted}. Fonte: Autor.}
\label{cod:python}
\end{codigo}





% ---------------------------------------------------
% SEÇÃO
% ---------------------------------------------------
\section{Figuras}


% ---------------------------------------------------
\subsection{Uma imagem}

\begin{figure}[H]
	\centering
	\includegraphics[width=0.3\linewidth]{Imagens/ufs_horizontal_positiva.png}
	\caption{Logo da Universidade Federal de Sergipe.}
	\label{fig:logo_ufs}
\end{figure}

Para referenciar uma figura, utiliza-se o comando \verb|\ref{<label>}|, onde o \texttt{<label>} foi previamente definido com \verb|\label{<label>}| dentro do ambiente da figura. No exemplo, o código \verb|Figura \ref{fig:logo_ufs}| gera a referência "Figura \ref{fig:logo_ufs}".

Exemplo: "De acordo com a Figura \ref{fig:logo_ufs}, temos que..."


% ---------------------------------------------------
\subsection{Duas imagens}

\begin{figure}[H]
    \centering
    
    \begin{subfigure}{0.3\textwidth}
        \centering
        \includegraphics[width=\linewidth]{Imagens/ufs_horizontal_positiva.png}
        \caption{}
        \label{fig:subimagem1}
    \end{subfigure}
    \hspace{1cm} % ou \hfill
    \begin{subfigure}{0.3\textwidth}
        \centering
        \includegraphics[width=\linewidth]{Imagens/DCOMP LOGO-AZUL NOVO-01.png}
        \caption{}
        \label{fig:subimagem2}
    \end{subfigure}
    
    \caption{Legenda geral para as subfiguras.}
    \label{fig:subfiguras_lado_a_lado}
\end{figure}

Também pode-se referenciar cada uma das duas imagens separadamente, como em: "Na Figura \ref{fig:subimagem1} e na Figura \ref{fig:subimagem2}", ou referenciar ela por completo: "Figura \ref{fig:subfiguras_lado_a_lado}".




% ---------------------------------------------------
% SEÇÃO
% ---------------------------------------------------
\section{Tabelas}


\begin{table}[H]
    \centering
    
    \begin{tabular}{cc}
        \toprule
        \textbf{Material}               & \textbf{Constante dielétrica \( k \)} \\
        \midrule
        Ar (1 atm)                      & 1,00054   \\
        Poliestireno                    & 2,6       \\
        Papel                           & 3,5       \\
        Óleo de transformador           & 4,5       \\
        Pirex                           & 4,7       \\
        Porcelana                       & 6,5       \\
        Água \( \SI{20}{\celsius} \)    & 80,4      \\
        Titânia \( \mathrm{TiO_2} \)    & 130       \\
        Titanato de estrôncio           & 310       \\
        \bottomrule
    \end{tabular}
    
    \caption{Propriedades de alguns dielétricos. Fonte: \cite{halliday}.}
    \label{tab:dieletricos}
\end{table}






% ---------------------------------------------------
% SEÇÃO
% ---------------------------------------------------
\section{Ambiente \texttt{blocks}}


% ---------------------------------------------------
\subsection{Texto corrido}

\begin{blocks}{Título}
    Lorem ipsum dolor sit amet, consectetuer adipiscing elit. Ut purus elit, vestibulum ut, placerat ac, adipiscing vitae, felis. Curabitur dictum gravida mauris. Nam arcu libero, nonummy eget, consectetuer id, vulputate a, magna. Donec vehicula augue eu neque.
\end{blocks}


% ---------------------------------------------------
\subsection{Teoremas}

\begin{codigo}[H]
\begin{minted}{tex}
\begin{teorema}{titulo}{label}
    <conteúdo>
\end{teorema}
%teste de comentário
\end{minted}
\caption{Código \LaTeX para ambiente \texttt{teorema}. Fonte: Autor.}
\label{cod:ambiente_teorema}
\end{codigo}

\begin{itemize}
    \item Exemplo com título
\end{itemize}
\begin{teorema}{Comprimento de curva parametrizada}{label}
Se uma curva \( C \) é descrita por equações paramétricas \( x=f(t) \) e \( y = g(t),\ \alpha \leq t \leq \beta \), onde \( f' \) e \( g' \) são contínuas em \( \left[ \alpha, \beta \right] \), então o comprimento de \( C \) é:
\[
L = \int_\alpha^\beta \sqrt{\left( \dfrac{dx}{dt} \right)^2 + \left( \dfrac{dy}{dt} \right)^2}\ dt
\]
\end{teorema}

\begin{itemize}
    \item Exemplo sem título: basta deixar o espaço \verb|{titulo}| vazio
\end{itemize}
\begin{teorema}{}{label2}
Se uma curva \( C \) é descrita por equações paramétricas \( x=f(t) \) e \( y = g(t),\ \alpha \leq t \leq \beta \), onde \( f' \) e \( g' \) são contínuas em \( \left[ \alpha, \beta \right] \), então o comprimento de \( C \) é:
\[
L = \int_\alpha^\beta \sqrt{\left( \dfrac{dx}{dt} \right)^2 + \left( \dfrac{dy}{dt} \right)^2}\ dt
\]
\end{teorema}

Para referenciar basta escrever \verb|De acordo com o Teorema \ref{teorema:<labe>}|, ficando:

De acordo com o Teorema \ref{teorema:label}




% ---------------------------------------------------
\subsection{Exemplos}

\begin{exemplo}
    Título do exemplo

    \textbf{\textit{Resolução:}}
    
    Início da resolução...\fimresolucao
\end{exemplo}






% ---------------------------------------------------
% SEÇÃO
% ---------------------------------------------------
\section{Plotagem de gráficos usando a biblioteca \texttt{tikz} e \texttt{pgfplots}}

\subsection{Função de uma variável}

\begin{figure}[H]
	\centering
	
	\begin{tikzpicture}
		\begin{axis}[
			width=9cm,
            %height=9cm,
			title={},
			%xlabel = \( x \),
			%ylabel = {\( f(x) \)},
			axis lines = middle,
			xmin=-4.5, xmax=4.5,
			ymin=-4.5, ymax=4.5,
			xtick={-4,-3,-2,-1,0,1,2,3,4},
			ytick={-4,-3,-2,-1,0,1,2,3,4},
			legend pos=north west,
			colormap/cool,			% COR
			]
			
			\addplot[		
			samples=200,
			color=red,
			domain=-2.5:2.5,
			%shader=interp,		% Habilita a interpolação de cores
			]
			{x^3 - 4*x};
			%\addlegendentry{\( x^3 - x \)}
		\end{axis}
	\end{tikzpicture}
	
	\label{fig:funcao_uma_variavel}
	\caption{Plotagem de \( f(x) = x^3 - x \).}
\end{figure}


% ---------------------------------------------------
\subsection{Funções paramétricas}

\begin{figure}[H]
	\centering
	
    \begin{tikzpicture}
        \begin{axis}[
            width=8cm,
            height=8cm,
            %x=1.0cm,
            %y=1.0cm,
            axis lines=middle,          % Eixos no centro (x=0, y=0)
            xlabel={$x$},               % Rótulo do eixo x
            ylabel={$y$},               % Rótulo do eixo y
            %title={$y^2 = x(x-3)^2$},    % Título do gráfico
            xmin=-2, xmax=6.5,
            ymin=-4.5, ymax=4.5,
            xtick={-2,-1,0,1,2,3,4,5,6},
			ytick={-4,-3,-2,-1,0,1,2,3,4},
            %grid=major,                 % Adiciona uma grade para melhor visualização
            %legend pos=north west,      % Posição da legenda
        ]
        \addplot[
            red,                    % Usa um nome de cor padrão, mais legível
            %line width=1.5pt,       % Largura de linha um pouco mais fina
            smooth,
            samples=150,            % Mais amostras para uma curva mais suave
            variable=t,
            domain=-2.2:2.2,        % Domínio otimizado para a área visível
        ]
        ({t^2}, {t^3-3*t});
        % \legend{$x=t^2$, $y=t^3-3t$} % Adiciona uma legenda à curva
        \end{axis}
    \end{tikzpicture}
    
    \label{fig:funcao_parametrica1}
	\caption{Plotagem de \( (x,y) = (t^2,\ t^3-3t), \quad -10 \leq 0 \leq 10 \).}
\end{figure}




\begin{figure}[H]
	\centering
	
	\begin{tikzpicture}
		\begin{axis}[
			width=8cm,
			title={},
			%xlabel = \( x \),
			%ylabel = {\( f(x) \)},
			%axis lines = middle,
			%xmin=-3, xmax=3,
			%ymin=-3, ymax=3,
			%xtick={-2,-1,0,1,2},
			%ytick={-2,-1,0,1,2},
			view={120}{30},
		]
		\addplot3[
			domain=0:5*pi,
			samples = 100,
			samples y=0,
		]
		({sin(deg(x))}, {cos(deg(x))}, {x});
		\end{axis}
	\end{tikzpicture}
	
	\label{fig:funcao_parametrica2}
	\caption{Plotagem de \( \langle x,y,z \rangle = \langle \sin (x),\ \cos (x),\ x \rangle \).}
\end{figure}


% ---------------------------------------------------
\subsection{Gráfico de duas variáveis}

O pacote não é muito bom para plotagem de gráficos de 2 variáveis.

\begin{figure}[H]
	\centering
	
	\begin{tikzpicture}
		\begin{axis}[
			height=10cm,
			width=8cm,
			title={Gráfico de \( z = x^2 + y^2 \)},
			xlabel={\(x\)},
			ylabel={\(y\)},
			zlabel={\(z\)},
			xmin=-5, xmax=5,
			ymin=-5, ymax=5,
			zmin=0, zmax=50, 			% Limites de Z ajustados
			ztick={0,10,20,30,40,50},
			view={120}{20} 				% Ajusta o ângulo de visão
			]
			\addplot3[
			surf,
			samples=50, 				% Espaço amostral
			domain=-5:5,
			]
			{x^2 + y^2};
		\end{axis}
	\end{tikzpicture}
	
	\label{fig:grafico3D}
	\caption{Plotagem de \( f(x,y) = x^2 + y^2 \).}
\end{figure}


\subsection{Malha diferenciada}

\begin{figure}[H]
	\centering
	
	\begin{tikzpicture}
		\begin{axis}[
			height=10cm,
			width=8cm,
			title={Gráfico de \( z = x^2 + y^2 \)},
			xlabel={\(x\)},
			ylabel={\(y\)},
			zlabel={\(z\)},
			%axis lines = middle,
			xmin=-5, xmax=5,
			ymin=-5, ymax=5,
			zmin=0, zmax=50, % Limites de Z ajustados
			ztick={0,10,20,30,40,50},
			view={120}{30} % Ajusta o ângulo de visão
			]
			\addplot3[
			surf,
            fill=white,  % Adicionada esta linha para o fundo branco
			samples=30, % Menos amostras para compilar mais rápido
			domain=-5:5,
			]
			{x^2 + y^2};
		\end{axis}
	\end{tikzpicture}
	
	\label{fig:grafico3D_mesh}
	\caption{Plotagem de \( f(x,y) = x^2 + y^2 \).}
\end{figure}


 % apague essa linha
    
    % ---- Descomente essas linhas para adicionar seus capítulos ----
    % ---- Lembre de criar um novo documento .tex na pasta "Capítulos"----
	\include{Capítulos/01_cap01.tex}
	% \include{Capítulos/02_cap02.tex}
	% \include{Capítulos/03_cap03.tex}
	
	% APÊNDICES
	\begin{appendices}
		%\include{Pos_textual/Apendice01.tex}
	\end{appendices}
	
	
	
	% -------------------------------------------------------
	% PÓS-TEXTUAL
	% -------------------------------------------------------
	\backmatter
	
	% ---- BIBLIOGRAFIA ----
	\nocite{*}
	\printbibliography[heading=bibintoc, title={Referências Bibliográficas}] % "bibintoc" inclui no sumário
	
	
\end{document}